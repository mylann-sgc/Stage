\documentclass[11pt,a4paper,oneside]{article}
\usepackage[utf8]{inputenc}
\usepackage[french]{babel}
\usepackage[T1]{fontenc}
\usepackage{graphicx}
\usepackage{charter}
\usepackage{hyperref}
\usepackage[left=2cm,right=2cm,top=2cm,bottom=2cm]{geometry}
\author{Mylann Dupuy}
\title{Rapport de stage}
\begin{document}
\maketitle
\newpage
\tableofcontents
\newpage
\section{Contexte}
Le centre Jean Bernard / Clinique Victor Hugo voudrait mettre un nouveau système de déploiement pour déployer diverses logiciels.
\section{Objectif(s)}
L'objectif est de déployer les logiciels utilisés couramment et de les mettre à jour. Le système doit fonctionner avant la fin du stage. 
\subsection{Cahier des charges}
bla bla
\\
\subsection{Contraintes}
Lors du déploiement, les postes sont constamment utilisé par le personnel du centre. Pour pouvoir intervenir, il faut appelé la personne présente sur le poste et intervenir à distance avec \textbf{TightVNC}.
\subsection{Matériels disponible}
\begin{itemize}
	\item \textbf{1 Ordinateur} pour administrer et surveiller OCS.
	\item \textbf{1 Serveur virtuel} sous 	\textbf{Debian 8.7}.
	\item \textbf{1 Serveur de stockage}  pour l'accès aux exécutables prévus pour les scripts. 

\end{itemize}
\section{Solutions}
\subsection{Comparaison}
En farfouillant sur Internet et aussi avec mes connaissances personnelles, j'ai recensé 5 solutions de déploiement mais dans notre contexte, il y a 2 solutions qui seront comparées \\ \\
\begin{tabular}{|p{3.1cm}|p{7cm}|p{7cm}|}
	\hline
	Solutions & Avantages & Inconvénients \\
	\hline
	\centering OCS Inventory NG  & \begin{itemize}
							\item Faible utilisation de la bande passante 
							\item Plugins pour GLPI
							\item Plugins pour GLPI
							\item Supervision des logiciels installé
							\item Logiciel libre disponible sous Windows Server / Client
							\item Inventaire complet des postes							
						\end{itemize} & \\
	\hline	
\end{tabular}
\subsection{Solution Choisie}
bla bla
\\
\subsection{Mise en \oe{}uvre}
bla bla
\\
\section{Principe de fonctionnement}
%Schéma
%Résumé d'utilisation d'OCS
% Utilisation du Module de déploiement
\subsection{Scripts utilisés}
bla bla
\section{Résultat Final}
bla bla
\subsection{Problèmes rencontrés}
bla bla
\section{Conclusion}
bla bla
\section{Annexes}
bla bla
\end{document}