\documentclass[11pt,a4paper,oneside]{article}
\usepackage[utf8]{inputenc}
\usepackage[french]{babel}
\usepackage[T1]{fontenc}
\usepackage{graphicx}
\usepackage{charter}
\usepackage{hyperref}
\usepackage[left=2cm,right=2cm,top=2cm,bottom=2cm]{geometry}
\author{Mylann Dupuy}
\title{Rapport de stage}
\begin{document}
\maketitle
\newpage
\tableofcontents
\newpage
\section{Contexte}
Le centre Jean Bernard / Clinique Victor Hugo voudrait mettre un nouveau système de d{\'e}ploiement pour d{\'e}ployer diverses logiciels.
\section{Objectif(s)}
L'objectif est de d{\'e}ployer les logiciels utilis{\'e}s couramment et de les mettre à jour. Le système doit fonctionner avant la fin du stage. 
\subsection{Cahier des charges}
bla bla
\\
\subsection{Contraintes}
Lors du d{\'e}ploiement, les postes sont constamment utilis{\'e} par le personnel du centre. Pour pouvoir intervenir, il faut appel{\'e} la personne pr{\'e}sente sur le poste et intervenir à distance avec \textbf{TightVNC}.
\subsection{Mat{\'e}riels disponible}
\begin{itemize}
	\item \textbf{1 Ordinateur} pour administrer et surveiller OCS.
	\item \textbf{1 Serveur virtuel} sous 	\textbf{Debian 8.7}.
	\item \textbf{1 Serveur de stockage}  pour l'accès aux ex{\'e}cutables pr{\'e}vus pour les scripts. 

\end{itemize}
\section{Solutions}
\subsection{Comparaison}
En farfouillant sur Internet et aussi avec mes connaissances personnelles, j'ai recens{\'e} 5 solutions de d{\'e}ploiement mais dans notre contexte, il y a 2 solutions qui seront compar{\'e}es \\ \\
%%%%%%%%%%%%%%%%%%%%%%%%%%%%%%%%%%%%%%%%%%%%%%%%%%%%%%%%%%%%%%%%%%%%%%%%%%%%%%%%%%%%%%%%%%%%%%%%%%%%%%%%%%%%%%%
\begin{tabular}{|p{3.1cm}|p{6.5cm}|p{6.5cm}|}
	\hline
	\centering Solutions : & \centering Avantages : & Inconv{\'e}nients : \\
	\hline
	%%%%%%%%%%%%%%%%%%%%%%%%%%%%%%%%%%%%%%%%%%%%%%%%%%%%%%%%%%%%%%%%%%%%%%%%%%%%%%%%%%%%%%%%%%%%%%%%%%%%%%%%%%%
	\centering OCS Inventory NG  & \begin{itemize}
							\item Faible utilisation de la bande passante 
							\item Plugins pour GLPI							
							\item Supervision des logiciels install{\'e}
							\item Logiciel libre disponible sous Windows Server / Client
							\item Inventaire complet des postes							
						\end{itemize} & \begin{itemize}
												\item Wiki non à jour  
												\item Paquets Debian en version 2.0.5																			\end{itemize} \\
	\hline
	%%%%%%%%%%%%%%%%%%%%%%%%%%%%%%%%%%%%%%%%%%%%%%%%%%%%%%%%%%%%%%%%%%%%%%%%%%%%%%%%%%%%%%%%%%%%%%%%%%%%%%%%%%%
	\centering WAPT  & \begin{itemize}
							\item Automatisation d'installation, MAJ et suppressions logiciels 
							\item Centralisation graphique du d{\'e}ploiement
							\item Facilit{\'e} pour les MAJ 
							\item Gestion des d{\'e}pendances
							\item Logiciel libre													
						\end{itemize} & \begin{itemize}
												\item Configuration à faire pour faire cohabiter WSUS et WAPT 
												\item Packages propre à WAPT (.wapt)
												\item Suite Microsoft Office non Disponible
												\item Supervision des logiciels install{\'e}
												\item Cr{\'e}ation de paquets + ou - complèxe
												\item Certains logiciels ne sont plus à jour			
										\end{itemize} \\
	\hline	
\end{tabular}
%%%%%%%%%%%%%%%%%%%%%%%%%%%%%%%%%%%%%%%%%%%%%%%%%%%%%%%%%%%%%%%%%%%%%%%%%%%%%%%%%%%%%%%%%%%%%%%%%%%%%%%%%%%%%%
\\ \\
Nous avons décidés de mettre en place OCS Inventory NG 2.3 car les logiciels qui doivent être déployer seront facilement mis à jour contrairement à WAPT.
\\
\subsection{Mise en \oe{}uvre}
bla bla
\\
\section{Principe de fonctionnement}
%Sch{\'e}ma
%R{\'e}sum{\'e} d'utilisation d'OCS
% Utilisation du Module de d{\'e}ploiement
\subsection{Scripts utilis{\'e}s}
bla bla
\section{R{\'e}sultat Final}
bla bla
\subsection{Problèmes rencontr{\'e}s}
bla bla
\section{Conclusion}
bla bla
\section{Annexes}
bla bla
\end{document}